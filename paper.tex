\def\paperversiondraft{draft}
\def\paperversionnormal{normal}

% If the paper version is set to 'normal' mode keep it,
% otherwise set it to 'draft' mode.
\ifx\paperversion\paperversionnormal
\else
  \def\paperversion{draft}
\fi

\documentclass[review, anonymous, acmsmall]{acmart}

\def\acmversionanonymous{anonymous}
\def\acmversionjournal{journal}
\def\acmversionnone{none}

\makeatletter
\if@ACM@anonymous
  \def\acmversion{anonymous}
\else
  \def\acmversion{journal}
\fi
\makeatother

\usepackage{colortbl}

% 'draftonly' environment
\usepackage{environ}
\ifx\paperversion\paperversiondraft
\newenvironment{draftonly}{}{}
\else
\NewEnviron{draftonly}{}
\fi

% Most PL conferences are edited by conference-publishing.com. Follow their
% advice to add the following packages.
%
% The first enables the use of UTF-8 as character encoding, which is the
% standard nowadays. The second ensures the use of font encodings that support
% accented characters etc. (Why should I use this?). The mictotype package
% enables certain features 'to­wards ty­po­graph­i­cal per­fec­tion
\usepackage[utf8]{inputenc}
\usepackage[T1]{fontenc}
\usepackage{microtype}

\usepackage{xargs}
\usepackage{lipsum}
\usepackage{xparse}
\usepackage{xifthen, xstring}
\usepackage{xspace}
\usepackage{marginnote}
\usepackage{etoolbox}
\usepackage[acronym,shortcuts]{glossaries}
\usepackage{amsmath}
\usepackage{thmtools} % required for autoref to lemmas
\usepackage{algorithm}
\usepackage[noend]{algpseudocode}
\usepackage{hyphenat}
\usepackage{mathtools}
\usepackage[shortcuts]{extdash}

%%%%%%%%%%%%%%%%%%%%%%%%%%%%%%%%%%%%%%%%%%%%%%%%%%%%%%%%%%%%%%%%%%%%%%%%%%%%%%%
% Broaden margins to make room for todo notes
%%%%%%%%%%%%%%%%%%%%%%%%%%%%%%%%%%%%%%%%%%%%%%%%%%%%%%%%%%%%%%%%%%%%%%%%%%%%%%%

\makeatletter
\patchcmd{\@addmarginpar}{\ifodd\c@page}{\ifodd\c@page\@tempcnta\m@ne}{}{}
\makeatother
\ifx\paperversion\paperversiondraft
  \makeatletter
  \if@ACM@journal
    \geometry{asymmetric}
    \paperwidth=\dimexpr \paperwidth + 3.5cm\relax
    \oddsidemargin=\dimexpr\oddsidemargin + 0cm\relax
    \evensidemargin=\dimexpr\evensidemargin + 0cm\relax
    \marginparwidth=\dimexpr \marginparwidth + 3cm\relax
    \setlength{\marginparwidth}{4.6cm}
    % This makeatletter box helps to move notes to the right
    \makeatletter
    \long\def\@mn@@@marginnote[#1]#2[#3]{%
      \begingroup
        \ifmmode\mn@strut\let\@tempa\mn@vadjust\else
          \if@inlabel\leavevmode\fi
          \ifhmode\mn@strut\let\@tempa\mn@vadjust\else\let\@tempa\mn@vlap\fi
        \fi
        \@tempa{%
          \vbox to\z@{%
            \vss
            \@mn@margintest
            \if@reversemargin\if@tempswa
                \@tempswafalse
              \else
                \@tempswatrue
            \fi\fi
            %\if@tempswa
              \rlap{%
                \if@mn@verbose
                  \PackageInfo{marginnote}{xpos seems to be \@mn@currxpos}%
                \fi
                \begingroup
                  \ifx\@mn@currxpos\relax\else\ifx\@mn@currxpos\@empty\else
                      \kern-\dimexpr\@mn@currxpos\relax
                  \fi\fi
                  \ifx\@mn@currpage\relax
                    \let\@mn@currpage\@ne
                  \fi
                  \if@twoside\ifodd\@mn@currpage\relax
                      \kern\oddsidemargin
                    \else
                      \kern\evensidemargin
                    \fi
                  \else
                    \kern\oddsidemargin
                  \fi
                  \kern 1in
                \endgroup
                \kern\marginnotetextwidth\kern\marginparsep
                \vbox to\z@{\kern\marginnotevadjust\kern #3
                  \vbox to\z@{%
                    \hsize\marginparwidth
                    \linewidth\hsize
                    \kern-\parskip
                    \marginfont\raggedrightmarginnote\strut\hspace{\z@}%
                    \ignorespaces#2\endgraf
                    \vss}%
                  \vss}%
              }%
          }%
        }%
      \endgroup
    }
    \makeatother
  \else
    \paperwidth=\dimexpr \paperwidth + 6cm\relax
    \oddsidemargin=\dimexpr\oddsidemargin + 3cm\relax
    \evensidemargin=\dimexpr\evensidemargin + 3cm\relax
    \marginparwidth=\dimexpr \marginparwidth + 3cm\relax
    \setlength{\marginparwidth}{4.6cm}
  \fi
  \makeatother
\fi

%%%%%%%%%%%%%%%%%%%%%%%%%%%%%%%%%%%%%%%%%%%%%%%%%%%%%%%%%%%%%%%%%%%%%%%%%%%%%%%
% Add \createtodoauthor command
%%%%%%%%%%%%%%%%%%%%%%%%%%%%%%%%%%%%%%%%%%%%%%%%%%%%%%%%%%%%%%%%%%%%%%%%%%%%%%%

\usepackage[textsize=tiny]{todonotes}
\usepackage[normalem]{ulem}

\makeatletter
\font\uwavefont=lasyb10 scaled 652
\newcommand\colorwave[1][blue]{\bgroup\markoverwith{\lower3\p@\hbox{\uwavefont\textcolor{#1}{\char58}}}\ULon}

\newcommand\highlight[2]{{\color{#1}{\colorwave[#1]{#2}}}}
\makeatother

\makeatletter
\newcommand\InFloat[2]{\ifnum\@floatpenalty<0\relax#1\else#2\fi}
\makeatother

\newboolean{inComment}
\setboolean{inComment}{false}

\ifx\paperversion\paperversiondraft
\newcommand\createtodoauthor[2]{
  \def\tmpdefault{emptystring}
  \expandafter\newcommand\csname #1\endcsname[2][\tmpdefault]{% comment to avoid spurious whitespace
    \ifthenelse{\boolean{inComment}}{% comment to avoid spurious whitespace
      \PackageError{paper-template}{Comments in comments not supported}{}% comment to avoid spurious whitespace
    }{}\setboolean{inComment}{true}% comment to avoid spurious whitespace
    \def\tmp{##1}% comment to avoid spurious whitespace
    \InFloat{
        \smash{
	  \marginnote{
	    \todo[inline,linecolor=#2,backgroundcolor=#2,bordercolor=#2]
	      {\textbf{#1 (Figure):} ##2}
          }
        }
    }{\ifthenelse{\equal{\tmp}{\tmpdefault}} % Is there text to highlight?
      {\todo[linecolor=#2,backgroundcolor=#2,bordercolor=#2]{\textbf{#1:} ##2}\ignorespaces}
      {\ifthenelse{\equal{##2}{}} % Is there a note?
        {\highlight{#2}{##1}}
        {\highlight{#2}{##1}\todo[linecolor=#2,backgroundcolor=#2,bordercolor=#2]
	  {\textbf{#1:} ##2}% comment to avoid spurious whitespace
	}% comment to avoid spurious whitespace
      }% comment to avoid spurious whitespace
    }% comment to avoid spurious whitespace
    \setboolean{inComment}{false}% comment to avoid spurious whitespace
  }
}
\else
\newcommand\createtodoauthor[2]{%
\expandafter\newcommand\csname #1\endcsname[2][]{##1}%
}%
\fi

%%%%%%%%%%%%%%%%%%%%%%%%%%%%%%%%%%%%%%%%%%%%%%%%%%%%%%%%%%%%%%%%%%%%%%%%%%%%%%%
% Add minted and support custom lexers
%%%%%%%%%%%%%%%%%%%%%%%%%%%%%%%%%%%%%%%%%%%%%%%%%%%%%%%%%%%%%%%%%%%%%%%%%%%%%%%
\usepackage{minted}
\usepackage{etoolbox}

\makeatletter
\ifcsdef{minted@optlistcl@quote}
{
\ifwindows
  \renewcommand{\minted@optlistcl@quote}[2]{%
    \ifstrempty{#2}{\detokenize{#1}}{\detokenize{#1="#2"}}}
\else
  \renewcommand{\minted@optlistcl@quote}[2]{%
    \ifstrempty{#2}{\detokenize{#1}}{\detokenize{#1='#2'}}}
\fi

% similar to \minted@def@optcl@switch
\newcommand{\minted@def@optcl@novalue}[2]{%
  \define@booleankey{minted@opt@g}{#1}%
    {\minted@addto@optlistcl{\minted@optlistcl@g}{#2}{}%
     \@namedef{minted@opt@g:#1}{true}}
    {\@namedef{minted@opt@g:#1}{false}}
  \define@booleankey{minted@opt@g@i}{#1}%
    {\minted@addto@optlistcl{\minted@optlistcl@g@i}{#2}{}%
     \@namedef{minted@opt@g@i:#1}{true}}
    {\@namedef{minted@opt@g@i:#1}{false}}
  \define@booleankey{minted@opt@lang}{#1}%
    {\minted@addto@optlistcl@lang{minted@optlistcl@lang\minted@lang}{#2}{}%
     \@namedef{minted@opt@lang\minted@lang:#1}{true}}
    {\@namedef{minted@opt@lang\minted@lang:#1}{false}}
  \define@booleankey{minted@opt@lang@i}{#1}%
    {\minted@addto@optlistcl@lang{minted@optlistcl@lang\minted@lang @i}{#2}{}%
     \@namedef{minted@opt@lang\minted@lang @i:#1}{true}}
    {\@namedef{minted@opt@lang\minted@lang @i:#1}{false}}
  \define@booleankey{minted@opt@cmd}{#1}%
      {\minted@addto@optlistcl{\minted@optlistcl@cmd}{#2}{}%
        \@namedef{minted@opt@cmd:#1}{true}}
      {\@namedef{minted@opt@cmd:#1}{false}}
}
\minted@def@optcl@novalue{customlexer}{-x}
}
{
}
\makeatother

\usepackage{tikz}
\usetikzlibrary{arrows}
\usetikzlibrary{shapes}

%%%%%%%%%%%%%%%%%%%%%%%%%%%%%%%%%%%%%%%%%%%%%%%%%%%%%%%%%%%%%%%%%%%%%%%%%%%%%%%
% Base style and command for \circled to print a colored circle
%%%%%%%%%%%%%%%%%%%%%%%%%%%%%%%%%%%%%%%%%%%%%%%%%%%%%%%%%%%%%%%%%%%%%%%%%%%%%%%
% Width is assured to be the same across all characters using the typewriter font which is monospaced
% Zeroing out the inner separator removes padding between content and node (inner sep).
% Zeroing out the outer separator removes space between the node border and its anchors (e.g., east).
% Minimum size was derived on experimentation and it may need adjustment when changing font style/size.
%
% There is no guarantee for the letter ascenders/descenders to baseline when set to char.base, hence adding \strut.
% which is an invisible vertical rule with the height and depth of the parentheses ( and ).
% It ensures that the line height in a line of text is at least as large as if it contained parentheses. 
\tikzset{
  circledstyle/.style={
    shape=circle,
    #1,
    font=\tt\small,
    inner sep=0pt,
    outer sep=0pt,
    minimum size=1.2em,
    text=black
  }
}

% define a base tikz node for circled commands accepting a fill colour and the node text as arguments
\DeclareRobustCommand{\circledbase}[3][]{%
    \tikz[baseline=(char.base)]{\node[circledstyle, fill=#2] (char) {#3\strut};}%
}

%%%%%%%%%%%%%%%%%%%%%%%%%%%%%%%%%%%%%%%%%%%%%%%%%%%%%%%%%%%%%%%%%%%%%%%%%%%%%%%
% Add refcheck support
%%%%%%%%%%%%%%%%%%%%%%%%%%%%%%%%%%%%%%%%%%%%%%%%%%%%%%%%%%%%%%%%%%%%%%%%%%%%%%%
\usepackage[norefs,nocites]{refcheck}

% make refcheck work with autoref
% based on: https://tex.stackexchange.com/questions/87610/making-refcheck-work-with-cleveref
\makeatletter
\newcommand{\refcheckize}[1]{%
  \expandafter\let\csname @@\string#1\endcsname#1%
  \expandafter\DeclareRobustCommand\csname relax\string#1\endcsname[1]{%
    \csname @@\string#1\endcsname{##1}\wrtusdrf{##1}}%
  \expandafter\let\expandafter#1\csname relax\string#1\endcsname
}
\makeatother

%%% add the reference commands we want refcheck to be aware of
\refcheckize{\autoref}


\ifx\acmversion\acmversionjournal
  %% Journal information (used by PACMPL format)
  %% Supplied to authors by publisher for camera-ready submission
  \acmJournal{PACMPL}
  \acmVolume{1}
  \acmNumber{1}
  \acmArticle{1}
  \acmYear{2017}
  \acmMonth{1}
  \acmDOI{10.1145/nnnnnnn.nnnnnnn}
  \startPage{1}
\else
  %% Conference information (used by SIGPLAN proceedings format)
  %% Supplied to authors by publisher for camera-ready submission
  \acmConference[PL'17]{ACM SIGPLAN Conference on Programming Languages}{January 01--03, 2017}{New York, NY, USA}
  \acmYear{2017}
  \acmISBN{978-x-xxxx-xxxx-x/YY/MM}
  \acmDOI{10.1145/nnnnnnn.nnnnnnn}
  \startPage{1}
\fi

%% Copyright information
%% Supplied to authors (based on authors' rights management selection;
%% see authors.acm.org) by publisher for camera-ready submission
%\setcopyright{none}             %% For review submission
%\setcopyright{acmcopyright}
%\setcopyright{acmlicensed}
%\setcopyright{rightsretained}
%\copyrightyear{2017}           %% If different from \acmYear


%% Bibliography style
\bibliographystyle{ACM-Reference-Format}
%% Citation style
%% Note: author/year citations are required for papers published as an
%% issue of PACMPL.
%\citestyle{acmauthoryear}  %% For author/year citations
%\citestyle{acmnumeric}     %% For numeric citations
%\setcitestyle{nosort}      %% With 'acmnumeric', to disable automatic
                            %% sorting of references within a single citation;
                            %% e.g., \cite{Smith99,Carpenter05,Baker12}
                            %% rendered as [14,5,2] rather than [2,5,14].
%\setcitesyle{nocompress}   %% With 'acmnumeric', to disable automatic
                            %% compression of sequential references within a
                            %% single citation;
                            %% e.g., \cite{Baker12,Baker14,Baker16}
                            %% rendered as [2,3,4] rather than [2-4].


\usemintedstyle{colorful}

% Newer versions of minted require the 'customlexer' argument for custom lexers
% whereas older versions require the '-x' to be passed via the command line.
\makeatletter
\ifcsdef{MintedExecutable}
{
  % minted v3
  \newminted[mlir]{tools/lexers/MLIRLexer.py:MLIRLexerOnlyOps}{mathescape}
  \newminted[xdsl]{tools/lexers/MLIRLexer.py:MLIRLexer}{mathescape, style=murphy}
  \newminted[lean4]{tools/lexers/Lean4Lexer.py:Lean4Lexer}{mathescape}
}
{
  \ifcsdef{minted@optlistcl@quote}
  {
    \newminted[mlir]{tools/lexers/MLIRLexer.py:MLIRLexerOnlyOps}{customlexer, mathescape}
    \newminted[xdsl]{tools/lexers/MLIRLexer.py:MLIRLexer}{customlexer, mathescape, style=murphy}
    \newminted[lean4]{tools/lexers/Lean4Lexer.py:Lean4Lexer}{customlexer, mathescape}
  }
  {
    \newminted[mlir]{tools/lexers/MLIRLexer.py:MLIRLexerOnlyOps -x}{mathescape}
    \newminted[xdsl]{tools/lexers/MLIRLexer.py:MLIRLexer -x}{mathescape, style=murphy}
    \newminted[lean4]{tools/lexers/Lean4Lexer.py:Lean4Lexer -x}{mathescape}
  }
}
\makeatother

% We use the following color scheme
%
% This scheme is both print-friendly and colorblind safe for
% up to four colors (including the red tones makes it not
% colorblind safe any more)
%
% https://colorbrewer2.org/#type=qualitative&scheme=Paired&n=4

\definecolor{pairedNegOneLightGray}{HTML}{cacaca}
\definecolor{pairedNegTwoDarkGray}{HTML}{827b7b}
\definecolor{pairedOneLightBlue}{HTML}{a6cee3}
\definecolor{pairedTwoDarkBlue}{HTML}{1f78b4}
\definecolor{pairedThreeLightGreen}{HTML}{b2df8a}
\definecolor{pairedFourDarkGreen}{HTML}{33a02c}
\definecolor{pairedFiveLightRed}{HTML}{fb9a99}
\definecolor{pairedSixDarkRed}{HTML}{e31a1c}

\createtodoauthor{grosser}{pairedOneLightBlue}
\createtodoauthor{authorTwo}{pairedTwoDarkBlue}
\createtodoauthor{authorThree}{pairedThreeLightGreen}
\createtodoauthor{authorFour}{pairedFourDarkGreen}
\createtodoauthor{authorFive}{pairedFiveLightRed}
\createtodoauthor{authorSix}{pairedSixDarkRed}

\newacronym{ir}{IR}{Intermediate Representation}

\graphicspath{{./images/}}

% Define macros that are used in this paper
%
% We require all macros to end with a delimiter (by default {}) to enusure
% that LaTeX adds whitespace correctly.
\makeatletter
\newcommand\requiredelimiter[2][########]{%
  \ifdefined#2%
    \def\@temp{\def#2#1}%
    \expandafter\@temp\expandafter{#2}%
  \else
    \@latex@error{\noexpand#2undefined}\@ehc
  \fi
}
\@onlypreamble\requiredelimiter
\makeatother

\newcommand\newdelimitedcommand[2]{
\expandafter\newcommand\csname #1\endcsname{#2}
\expandafter\requiredelimiter
\csname #1 \endcsname
}

\newdelimitedcommand{toolname}{Tool}

\usepackage{booktabs}
\newcommand{\ra}[1]{\renewcommand{\arraystretch}{#1}}

\usepackage[verbose]{newunicodechar}
\newunicodechar{₁}{\ensuremath{_1}}
\newunicodechar{₂}{\ensuremath{_2}}
\newunicodechar{∀}{\ensuremath{\forall}}
\newunicodechar{α}{\ensuremath{\alpha}}
\newunicodechar{β}{\ensuremath{\beta}}

% \circled command to print a colored circle.
% \circled{1} pretty-prints "(1)"
% This is useful to refer to labels that are embedded within figures.
\DeclareRobustCommand{\circled}[2][]{%
    \ifthenelse{\isempty{#1}}%
        {\circledbase{pairedOneLightBlue}{#2}}%
        {\autoref{#1}: \hyperref[#1]{\circledbase{pairedOneLightBlue}{#2}}}%
}

% listings don't write "Listing" in autoref without this.
\providecommand*{\listingautorefname}{Listing}
\renewcommand{\sectionautorefname}{Section}
\renewcommand{\subsectionautorefname}{Section}
\renewcommand{\subsubsectionautorefname}{Section}

\begin{document}

%% Title information
\title[Short Title]{Full Title}       %% [Short Title] is optional;
                                      %% when present, will be used in
                                      %% header instead of Full Title.
\subtitle{Subtitle}                   %% \subtitle is optional


%% Author information
%% Contents and number of authors suppressed with 'anonymous'.
%% Each author should be introduced by \author, followed by
%% \authornote (optional), \orcid (optional), \affiliation, and
%% \email.
%% An author may have multiple affiliations and/or emails; repeat the
%% appropriate command.
%% Many elements are not rendered, but should be provided for metadata
%% extraction tools.
\author{First1 Last1}
\authornote{with author1 note}          %% \authornote is optional;
                                      %% can be repeated if necessary
\orcid{nnnn-nnnn-nnnn-nnnn}             %% \orcid is optional
\affiliation{
  \position{Position1}
  \department{Department1}              %% \department is recommended
  \institution{Institution1}            %% \institution is required
  \streetaddress{Street1 Address1}
  \city{City1}
  \state{State1}
  \postcode{Post-Code1}
  \country{Country1}
}
\email{first1.last1@inst1.edu}          %% \email is recommended

\author{First2 Last2}
\authornote{with author2 note}          %% \authornote is optional;
                                      %% can be repeated if necessary
\orcid{nnnn-nnnn-nnnn-nnnn}             %% \orcid is optional
\affiliation{
  \position{Position2a}
  \department{Department2a}             %% \department is recommended
  \institution{Institution2a}           %% \institution is required
  \streetaddress{Street2a Address2a}
  \city{City2a}
  \state{State2a}
  \postcode{Post-Code2a}
  \country{Country2a}
}
\email{first2.last2@inst2a.com}         %% \email is recommended
\affiliation{
  \position{Position2b}
  \department{Department2b}             %% \department is recommended
  \institution{Institution2b}           %% \institution is required
  \streetaddress{Street3b Address2b}
  \city{City2b}
  \state{State2b}
  \postcode{Post-Code2b}
  \country{Country2b}
}
\email{first2.last2@inst2b.org}         %% \email is recommended

\begin{abstract}
% An abstract should consist of six main sentences:
%  1. Introduction. In one sentence, what’s the topic?
%  2. State the problem you tackle.
%  3. Summarize (in one sentence) why nobody else has adequately answered the research question yet.
%  4. Explain, in one sentence, how you tackled the research question.
%  5. In one sentence, how did you go about doing the research that follows from your big idea.
%  6. As a single sentence, what’s the key impact of your research?

% (http://www.easterbrook.ca/steve/2010/01/how-to-write-a-scientific-abstract-in-six-easy-steps/)

  % generated by lipsum
  Curabitur dictum gravida mauris.
  Nam arcu libero, nonummy eget, consectetuer id, vulputate a, magna.
  Donec vehicula augue eu neque.
  Pellentesque habitant morbi tristique senectus et netus et malesuada fames ac turpis egestas.
  Cras viverra metus rhoncus sem.

  Our \emph{key} insight defines the mass-energy equivalence relationship as: $E = mc^2$.
  We prototype our approach in \toolname{}.

  % generated by lipsum
  Curabitur dictum gravida mauris.
  Nam arcu libero, nonummy eget, consectetuer id, vulputate a, magna.
  Donec vehicula augue eu neque.
  Pellentesque habitant morbi tristique senectus et netus et malesuada fames ac turpis egestas.
  Cras viverra metus rhoncus sem.
\end{abstract}

% Only add ACM notes and keywords in camera ready version
% Drop citations and footnotes in draft and blind mode.
\ifx\acmversion\acmversionanonymous
\settopmatter{printacmref=false} % Removes citation information below abstract
\renewcommand\footnotetextcopyrightpermission[1]{} % removes footnote with conference information in first column
\fi
\ifx\acmversion\acmversionjournal
%% 2012 ACM Computing Classification System (CSS) concepts
%% Generate at 'http://dl.acm.org/ccs/ccs.cfm'.
\begin{CCSXML}
<ccs2012>
<concept>
<concept_id>10011007.10011006.10011008</concept_id>
<concept_desc>Software and its engineering~General programming languages</concept_desc>
<concept_significance>500</concept_significance>
</concept>
<concept>
<concept_id>10003456.10003457.10003521.10003525</concept_id>
<concept_desc>Social and professional topics~History of programming languages</concept_desc>
<concept_significance>300</concept_significance>
</concept>
</ccs2012>
\end{CCSXML}

\ccsdesc[500]{Software and its engineering~General programming languages}
\ccsdesc[300]{Social and professional topics~History of programming languages}
%% End of generated code

%% Keywords
%% comma separated list
\keywords{keyword1, keyword2, keyword3}
\fi

%% \maketitle
%% Note: \maketitle command must come after title commands, author
%% commands, abstract environment, Computing Classification System
%% environment and commands, and keywords command.
\maketitle

\section{Introduction}

Of this super cool article \ldots
\grosser{A simple comment refering to a particular location}

If there is \authorTwo[some text we want to refer to specifically]{this looks
particularly interesting} we can refer to this text from within our comments.

And here we only mark \authorTwo[text]{} that is \authorTwo[interesting]{}, but do not
actually comment.

Now we have a sequence of comments,
\authorThree{Comment 3}
\authorFour{Comment 4}
\authorFive{Comment 5}
\authorSix{Comment 6}
which should not introduce overlong white space sequences.

Sometimes it is also important to \emph{emphasize} text to understand if it is
underlined or printed in italic.

We sometimes use macros to state that a tool as the name \toolname{} in a
flexible way.

\lipsum[1-3]

\begin{figure}
% Link to figure
%
% https://docs.google.com/drawings/d/1juKp43D3rLC-luBQPwQZ_wCnDK2S_6C1k6USV0wKE0g/edit?usp=sharing
\includegraphics[width=\columnwidth]{overview.pdf}
\caption{Our key idea visualized}
\grosser{Replace this figure with your own drawing.}
\end{figure}

\vspace{.5em}
\noindent
Our contributions are:
\grosser{Always state your contributions explicitly: (A) this makes it
easy for the reader to understand what novelty is presented, and (B)
these contributions help you to focus. In particular, the objective of
the remaining paper should be to support the claims stated here.
}
\begin{itemize}
	\item Contribution 1 (\autoref{sec:implementation})
	\item Contribution 2
	\item Contribution 3
	\item Contribution 4
\end{itemize}

\section{Our New Idea}

\lipsum[1-3]

\section{Implementation}
\label{sec:implementation}

\section{Related Work}

\lipsum[1-3]

\grosser{Related work should always be at the end of the document,
         as it otherwise becomes an obstacle your reader must
	 overcome before reaching your idea. For details see:
	 \url{https://www.microsoft.com/en-us/research/academic-program/write-great-research-paper/}
}

\section{Conclusion}
\lipsum[1]

%% Acknowledgments
\begin{acks}                            %% acks environment is optional
                                        %% contents suppressed with 'anonymous'
  %% Commands \grantsponsor{<sponsorID>}{<name>}{<url>} and
  %% \grantnum[<url>]{<sponsorID>}{<number>} should be used to
  %% acknowledge financial support and will be used by metadata
  %% extraction tools.
  This material is based upon work supported by the
  \grantsponsor{GS100000001}{National Science
    Foundation}{http://dx.doi.org/10.13039/100000001} under Grant
  No.~\grantnum{GS100000001}{nnnnnnn} and Grant
  No.~\grantnum{GS100000001}{mmmmmmm}.  Any opinions, findings, and
  conclusions or recommendations expressed in this material are those
  of the author and do not necessarily reflect the views of the
  National Science Foundation.
\end{acks}

%% Bibliography
\bibliography{references}

\end{document}
